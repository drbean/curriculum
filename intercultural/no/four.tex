\documentclass[a4paper]{article}
\usepackage[T1]{fontenc}
% \usepackage[absolute,noshowtext,showboxes]{textpos}
\usepackage[absolute,showboxes]{textpos}
% \textblockorigin{0.10cm}{1.00cm}
\textblockorigin{0.00cm}{0.00cm} %HPLaserJet5000LE
\usepackage{texdraw}
\pagestyle{empty}
\setlength{\unitlength}{1cm}

\newcommand{\myIdentifier}[0]{
Cultural differences on saying 'no'
}

\newcommand{\myAcontent}[0]{
Chen and Smith lack 1. the ability/desire to avoid ethnocentrism and to see their conversation from the viewpoint of the other person. 2. knowledge of Western or Asian rules about explicitness/implicitness. 3. Smith forgets Westerners also don't like to say, 'No,' because it hurts people's feelings. Chen forgets that lack of clear answers also sometimes causes misunderstanding in East Asia.
}

\newcommand{\myBcontent}[0]{
4. Smith isn't able to ask the right questions to find out how Chen would say, 'No'. He can't learn about ways Chinese disagree. Chen perhaps knows that Westerners like clear answers, but can't use what he knows. He can't give clear answers in English. 5. Smith is not aware of his own ideas about disagreement. Chen perhaps has an opinion of Western rules, but has not evaluated his own ideas.
}

\newcommand{\myCcontent}[0]{
You are Smith. Chen had objections to your plan. He saw problems, which you didn't think were very important. But you gave reasons why you thought these problems could be overcome. You both then stopped talking. You thought this meant that he agreed the problems could be overcome, and that he accepted your plan. But this was a misunderstanding.
}

\newcommand{\myDcontent}[0]{
You are Chen. You think Smith's plan is not good. You did not want to challenge him. You told him of some little problems with the plan. He argued with you about those problems. You think he did this to save face, because you think his arguments were not very good. Anyway you both stopped talking. You thought this meant he realized that the plan was not good. But this was a misunderstanding.
}

\newcommand{\mycard}[5]{%
	\vspace{0.1cm}
	\small #1 #2
	\par
	\parbox[t][6.7cm][c]{9.5cm}{%
	\hspace{0.1cm} \Large#3\\
	\normalsize#4 #5
	}
}

\begin{document}
\fontfamily{hlst}\fontseries{b}\fontshape{n}\selectfont
% \begin{picture}(15,20)(+4.8,-22.05)
% \begin{tabular}[t]{*{2}{|p{10.05cm}}|}

\begin{textblock}{8}(0,0)
\textblocklabel{picture1}
\mycard{}{\myIdentifier}{\parbox{9.0cm}{A:
\myAcontent
}}{}{} 
\end{textblock}

\begin{textblock}{8}(8,0)
\textblocklabel{picture2}
\mycard{}{\myIdentifier}{\parbox{9.0cm}{B:
\myBcontent
}}{}{} 
\end{textblock}

\begin{textblock}{8}(0,4)
\textblocklabel{picture3}
\mycard{}{\myIdentifier}{\parbox{9.0cm}{C:
\myCcontent
}}{}{} 
\end{textblock}

\begin{textblock}{8}(8,4)
\textblocklabel{picture4}
\mycard{}{\myIdentifier}{\parbox{9.0cm}{D:
\myDcontent
}}{}{} 
\end{textblock}

\begin{textblock}{8}(0,8)
\textblocklabel{picture5}
\mycard{}{\myIdentifier}{\parbox{9.0cm}{A:
\myAcontent
}}{}{} 
\end{textblock}

\begin{textblock}{8}(8,8)
\textblocklabel{picture6}
\mycard{}{\myIdentifier}{\parbox{9.0cm}{B:
\myBcontent
}}{}{} 
\end{textblock}

\begin{textblock}{8}(0,12)
\textblocklabel{picture7}
\mycard{}{\myIdentifier}{\parbox{9.0cm}{C:
\myCcontent
}}{}{} 
\end{textblock}

\begin{textblock}{8}(8,12)
\textblocklabel{picture8}
\mycard{}{\myIdentifier}{\parbox{9.0cm}{D:
\myDcontent
}}{}{} 
\end{textblock}

\end{document}


