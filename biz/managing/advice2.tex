\documentclass[a4paper]{article}
\usepackage[T1]{fontenc}
% \usepackage[absolute,noshowtext,showboxes]{textpos}
\usepackage[absolute,showboxes]{textpos}
% \textblockorigin{0.10cm}{1.00cm}
\textblockorigin{0.00cm}{0.10cm} % HPLaserJet2300
% \textblockorigin{0.00cm}{0.00cm} %HPLaserJet5000LE
\usepackage{texdraw}
\pagestyle{empty}
\setlength{\unitlength}{1cm}

\newcommand{\myIdentifier}[0]{
advice 2
}

\newcommand{\myAcontent}[0]{
To help the 2 different teams work together, Nicholson says the manager must find things which the 2 teams agree about, for example, the desire to do a good job. Then the manager can help them agree about other things, for example, the need to work together. The manager needs to find people who think differently to the others. These people might agree with the other team.
}

\newcommand{\myBcontent}[0]{
Nicholson says managers must examine their own and the other people's motivation. Todd, the marketing manager needs to understand why Oliver wants to be a super salesman. But he also needs to understand why he wants to control Oliver. He needs to understand why Oliver thinks differently than him. He needs to ask himself: Am I wrong? Is Oliver right.
}

\newcommand{\myCcontent}[0]{
Nicholson says the Muller sales team aren't able to see their job of marketing as part of the company system. They just think about selling the product and not about the connection with production. They don't realize that the company is a system. Everyone needs to cooperate. Everyone needs to have all the information, so that the system works well.
}

\newcommand{\myDcontent}[0]{
Nicholson says the Peterson sales team understands the systems view. It understands it is part of a system. It also understands the company is part of a larger system with customers and other companies. It may not understand it is the most important part of the system. It must understand its contacts with customers make the company a success or a failure.
}

\newcommand{\mycard}[5]{%
	\vspace{0.1cm}
	\small #1 #2
	\par
	\parbox[t][6.7cm][c]{9.5cm}{%
	\hspace{0.1cm} \Large#3\\
	\normalsize#4 #5
	}
}

\begin{document}
\fontfamily{hlst}\fontseries{b}\fontshape{n}\selectfont
% \begin{picture}(15,20)(+4.8,-22.05)
% \begin{tabular}[t]{*{2}{|p{10.05cm}}|}

\begin{textblock}{8}(0,0)
\textblocklabel{picture1}
\mycard{}{\myIdentifier}{\parbox{9.0cm}{A:
\myAcontent
}}{}{} 
\end{textblock}

\begin{textblock}{8}(8,0)
\textblocklabel{picture2}
\mycard{}{\myIdentifier}{\parbox{9.0cm}{B:
\myBcontent
}}{}{} 
\end{textblock}

\begin{textblock}{8}(0,4)
\textblocklabel{picture3}
\mycard{}{\myIdentifier}{\parbox{9.0cm}{C:
\myCcontent
}}{}{} 
\end{textblock}

\begin{textblock}{8}(8,4)
\textblocklabel{picture4}
\mycard{}{\myIdentifier}{\parbox{9.0cm}{D:
\myDcontent
}}{}{} 
\end{textblock}

\begin{textblock}{8}(0,8)
\textblocklabel{picture5}
\mycard{}{\myIdentifier}{\parbox{9.0cm}{A:
\myAcontent
}}{}{} 
\end{textblock}

\begin{textblock}{8}(8,8)
\textblocklabel{picture6}
\mycard{}{\myIdentifier}{\parbox{9.0cm}{B:
\myBcontent
}}{}{} 
\end{textblock}

\begin{textblock}{8}(0,12)
\textblocklabel{picture7}
\mycard{}{\myIdentifier}{\parbox{9.0cm}{C:
\myCcontent
}}{}{} 
\end{textblock}

\begin{textblock}{8}(8,12)
\textblocklabel{picture8}
\mycard{}{\myIdentifier}{\parbox{9.0cm}{D:
\myDcontent
}}{}{} 
\end{textblock}

\end{document}


